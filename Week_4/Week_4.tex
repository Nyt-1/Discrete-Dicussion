\documentclass[12pt]{article}
 \usepackage[margin=1in]{geometry} 
\usepackage{amsmath,amsthm,amssymb,amsfonts, enumitem, fancyhdr, color, comment, graphicx, environ}
\usepackage{booktabs}
\usepackage[colorinlistoftodos]{todonotes}
\usepackage{algorithm}
\usepackage{algpseudocode}
\usepackage{tikz}
\usepackage{subcaption}
\usepackage{float}
\usetikzlibrary{arrows.meta, automata, positioning}

\usepackage{listings}
\usepackage{xcolor}
\lstset{
  basicstyle=\ttfamily\small,
  backgroundcolor=\color{gray!10},
  frame=single,
  breaklines=true,
  keywordstyle=\color{blue},
  commentstyle=\color{gray},
  stringstyle=\color{red},
  showstringspaces=false,
  numbers=left,            
  numberstyle=\tiny\color{gray}, 
  numbersep=10pt 
}

%%%%%%%%%%%%%%%%%%%%%%%%%%%%%%%%%%%%%%%%%%%%%%%%%%%%%%%%%%%%%%%%%%%%%%%%
\pagestyle{fancy}
\setlength{\headheight}{65pt}
\newenvironment{problem}[2][Problem]{\begin{trivlist}
\item[\hskip \labelsep {\bfseries #1}\hskip \labelsep {\bfseries #2.}]}{\end{trivlist}}
\newenvironment{sol}
    {\emph{Solution:}
    }
    {
    \qed
    }
\specialcomment{com}{ \color{blue} \textbf{Comment:} }{\color{black}} %for instructor comments while grading
\NewEnviron{probscore}{\marginpar{ \color{blue} \tiny Problem Score: \BODY \color{black} }}

\newcounter{subproblem}
% \renewcommand{\thesubproblem}{\alph{subproblem}} % letters 
\renewcommand{\thesubproblem}{\arabic{subproblem}} % numbers
\newenvironment{subprob}[1][]{
  \refstepcounter{subproblem}
  \begin{trivlist}
  \item[\hskip \labelsep {\bfseries (\thesubproblem)}]
}{
  \end{trivlist}
}
\newenvironment{subsol}
    {\emph{Solution:}
    }
    {
    \qed
    }

\newtheorem{theorem}{Theorem}
\newtheorem{lemma}{Lemma}
\newtheorem{proposition}{Proposition}
\newtheorem{corollary}{Corollary}

% numbered within sections (optional)
% \newtheorem{theorem}{Theorem}[section]
% \newtheorem{lemma}[theorem]{Lemma} % shares theorem counter

%%%%%%%%%%%%%%%%%%%%%%%%%%%%%%%%%%%%%%%%%%%%%%%%%%%%%%%%%%%%%%%%%%%%%%%%%%%%%%%%%
\setlength {\marginparwidth }{2cm}
\newcommand{\nextline}{\\[0.2em]}


\newtheorem*{remark}{Remark}

\newcommand{\R}{\mathbb{R}}
\newcommand{\N}{\mathbb{N}}
\newcommand{\Q}{\mathbb{Q}}
\newcommand{\Z}{\mathbf{Z}}
\newcommand{\st}{\text{ s.t }}
\newcommand{\tand}{\text{ and }}
\newcommand{\tor}{\text{ or }}
\newcommand{\bigO}[1]{\mathcal{O}\left(#1\right)}
\newcommand{\fP}{\mathbb{P}}
\newcommand{\E}{\mathbb{E}}

%%%% In most cases you won't need to edit anything above this line %%%%

\begin{document}

\hfill CMSC 27100 Practice Problems

\hfill Week \#4

\section*{Equivalence Classes and CRT}
\begin{enumerate}
    \item Prove that $\forall k \geq 1, ~\exists N \in \Z \st N \pmod{M = \prod_{i = 1}^{k} p_i}$ satisfies the following by constructing a $N$ such that:
        \begin{enumerate}
            \item $N \equiv -1 \pmod{p_i} ~\forall i \in [k]$
            \item $N \equiv 2^i \pmod{p_i} ~\forall i \in [k]$
        \end{enumerate}
\end{enumerate}

\section*{Induction}
\begin{enumerate}
    \item Prove that every integer $n > 1$ can be written unqiely as the product of primes. Prove existance and uniqueness.
    \item Let $S = \{6a + 9b + 20c \st a,b,c \in Z_{\geq 0}\}$. Prove that any integer $n \geq 44$ is in $S$.
\end{enumerate}

\section*{Wilson's Theorem}
\begin{enumerate}
    \item For what $n$ does $(n-1)! \equiv -1 \pmod{n}$ hold? Prove this as an iff relationship.
\end{enumerate}

\section*{Fermat's Little Theorem}
\begin{enumerate}
    \item Compute $7^{529} \pmod{23}$
    \item Find $37^{-1} \pmod{101}$
\end{enumerate}
\end{document}
