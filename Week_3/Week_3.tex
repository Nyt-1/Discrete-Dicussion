\documentclass[12pt]{article}
 \usepackage[margin=1in]{geometry} 
\usepackage{amsmath,amsthm,amssymb,amsfonts, enumitem, fancyhdr, color, comment, graphicx, environ}
\usepackage{booktabs}
\usepackage{}
\usepackage[colorinlistoftodos]{todonotes}
\usepackage{algorithm}
\usepackage{algpseudocode}
\usepackage{amsmath}
\usepackage{tikz}
\usepackage{subcaption}
\usetikzlibrary{arrows.meta, automata, positioning}


\pagestyle{fancy}
\setlength{\headheight}{65pt}
\newenvironment{problem}[2][Problem]{\begin{trivlist}
\item[\hskip \labelsep {\bfseries #1}\hskip \labelsep {\bfseries #2.}]}{\end{trivlist}}
\newenvironment{sol}
    {\emph{Solution:}
    }
    {
    \qed
    }
\specialcomment{com}{ \color{blue} \textbf{Comment:} }{\color{black}} %for instructor comments while grading
\NewEnviron{probscore}{\marginpar{ \color{blue} \tiny Problem Score: \BODY \color{black} }}

\newcounter{subproblem}
% \renewcommand{\thesubproblem}{\alph{subproblem}} % letters 
\renewcommand{\thesubproblem}{\arabic{subproblem}} % numbers
\newenvironment{subprob}[1][]{
  \refstepcounter{subproblem}
  \begin{trivlist}
  \item[\hskip \labelsep {\bfseries (\thesubproblem)}]
}{
  \end{trivlist}
}
\newenvironment{subsol}
    {\emph{Solution:}
    }
    {
    \qed
    }
\newtheorem{theorem}{Theorem}
\newtheorem{lemma}{Lemma}
\newtheorem{proposition}{Proposition}
\newtheorem{corollary}{Corollary}
%%%%%%%%%%%%%%%%%%%%%%%%%%%%%%%%%%%%%%%%%%%%%%%%%%%%%%%%%%%%%%%%%%%%%%%%%%%%%%%%%
\setlength {\marginparwidth }{2cm}


\newtheorem*{remark}{Remark}

\newcommand{\R}{\mathbb{R}}
\newcommand{\N}{\mathbb{N}}
\newcommand{\Q}{\mathbb{Q}}
\newcommand{\Z}{\mathbf{Z}}
\newcommand{\st}{\text{ s.t }}
\newcommand{\bigO}[1]{\mathcal{O}\left(#1\right)}
\newcommand{\fP}{\mathbb{P}}
\newcommand{\E}{\mathbb{E}}


\usepackage{listings}
\usepackage{xcolor}
\lstset{
  basicstyle=\ttfamily\small,
  backgroundcolor=\color{gray!10},
  frame=single,
  breaklines=true,
  keywordstyle=\color{blue},
  commentstyle=\color{gray},
  stringstyle=\color{red},
  showstringspaces=false,
  numbers=left,            
  numberstyle=\tiny\color{gray}, 
  numbersep=10pt 
}


\textwidth 7.0 truein
\oddsidemargin -0.25in   %left-hand edge
\evensidemargin -0.5 truein  %right-hand edge
\topmargin -0.85in      %top of paper to top of head, pulls whole unit
\textheight 9.5in

%%%% In most cases you won't need to edit anything above this line %%%%

\begin{document}
\hfill CMSC 27100 Practice Problems

\hfill Week \#3

%%%%%%%%%%%%%%%%%%%%%%%%%%%%%%%%%%%%%%%%%%%%%%%%%%%%%%%%%%%%%%%%%%%%%%%%%%%%%%%%%
\section*{Division Theorem}
\begin{theorem}[Division Algorithm]
For every integer $n$ and every integer $d>0$, there exist unique integers $q$ and $r$ such that
\[
n = dq + r \quad \text{and} \quad 0 \le r < d.
\]
\end{theorem}
\begin{proof} 

    \textbf{Existance} \\
    Consider the set $A = \{n-dq |q \in \Z\}$. Since $d \neq 0$, $A$ must have a non-negative number. By ordering, we know there must exist a smallest non-negative number, We call this $r$. $r < d$ becuase otherwise, $r = n - dq = d + m$ and so $m = n - d(q+1)$ so $m \in A,~ m < r$, a contradiction. 

    \textbf{Uniqueness}
    \begin{align*}
        n &= dq_1 + r_1 = dq_2 + r_2 \\
        0 &= d(q_1 - q_2) + (r_1 - r_2) \\
        r_2 - r_1 &= d(q_1 - q_2) \\
    \end{align*}

    $d | r_2 - r_1, ~ 0 \leq r_2 < d ~\Rightarrow~r_2 - r_1 = 0 ~\Rightarrow ~r_2 = r_1 ~\Rightarrow ~q_2 = q_1$
\end{proof}

\begin{enumerate}
    \item Prove $\forall n \in \Z,$ let $r$ be the remainder of $n$ divided by $d$, prove that $n = r \pmod{d}$
    \item Let $d > 0$, prove $d | (n-m) \iff n$ and $m$ have the same remainder divided by d
\end{enumerate}

\section*{Euclid's Algorithm}
If $d=gcd(a,b), b\neq 0 \text{ and } r = a \mod b, \text{ then } d = gcd(b,r)$

\begin{enumerate}
    \item Compute $gcd(105, 252)$
    \item Prove $gcd(a,b) = gcd(a, b - ka)$ for any integer $k$. Use this fact to compute $gcd(98765, 43210)$
\end{enumerate}

\section*{Modular Arthemtic}
\begin{enumerate}
    \item Compute $7^{2001} \pmod{1000}$
    \item Find $x \st 35x = 10 \pmod{50}$
\end{enumerate}



\end{document}
